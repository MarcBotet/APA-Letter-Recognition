\section{Conclusions}
A l'hora enfrontar la complexitat de la predicció del reconeixement de lletres es pot concloure que hi ha uns quants models que aconsegueixen sobrepassar el 80\% de precisió que s'aconseguia en l'article\cite{frey1991letter} mencionat. Clarament es pot inferir que els mètodes no lineals donen millors resultats per norma general que els lineals, encara que, com s'ha vist anteriorment, hi ha alguns mètodes no lineals que donen resultants molt precisos superiors al 90\% com és el cas de SVM quadràtic.

Per la naturalesa del nostre conjunt de dades i la seva simplicitat, que no necessita cap tipus de preprocessat potser ens ha limitat a l'hora de poder explotar al màxim tots els coneixements estudiats en classe com seria el cas de reducció de dimensionalitat i tractament de valors anormals i/o faltants.

En un principi es van mirar altres mètodes per treballar com Naive Bayes, però pels seus mal resultats i poca comoditat per treballar amb aquests models es van acabar descartant.

Aquest treball ens ha servit per introduir-nos en el món de l'aprenentatge automàtic, un món del qual n'hem sentit a parlar molt. Ens ha permès donar-nos compte de les dificultats i limitacions d'aquesta disciplina com, per exemple, la gran quantitat de poder computacional que es necessita per treballar en conjunts de dades molt grans, ja que en els nostres ordinadors tardava hores només tenint vint mil exemples. Però tanmateix ens ha fet reflexionar sobre la multitud d'aplicacions en el món real que poden estalviar-nos una gran quantitat de temps i millorar la nostra vida.

Una possible extensió d'aquest treball seria estudiar el comportament de totes les variables a l'hora de predir una lletra, i estudiar quina de les 16 variables és més útil i quina ho és menys per a arribar a la conclusió. També es podria comparar els resultats amb nous enfocaments de reconeixement de lletres que no usen característiques concretes com les que teníem en el nostre conjunt de dades, sinó que la seva entrada són tots els píxels de la imatge.